\documentclass[10pt,a4paper]{book}
\usepackage[italian]{babel}
\usepackage[utf8]{inputenc}
\usepackage{amsmath}
\usepackage{amsfonts}
\usepackage{amssymb}
\usepackage{graphicx}
\usepackage{float} 
\begin{document}
\chapter{Introduzione}

\section{Tipi di ragionamenti}
\begin{itemize}
\item \textbf{Ragionamento Deduttivo}: da ricondurre a \textit{Aristotele}(384 a.C.-322a.C.).\\
Da una regola e da un caso ottengo un risultato. \`E il ragionamento che permette la dimostrazione dei teoremi, ed \`e stata la base dei AI degli anni '90.\\\\
Esempio\\
REGOLA: "Tutti gli uomini sono mortali."\\
CASO: "Socrate \`e un uomo."\\
RISULTATO: "Socrate \`e mortale."
\item \textbf{Ragionamento Induttivo}:da ricondurre a \textit{Bacon} (1561-1626).\\
Avviene un'operazione di generalizzazione. Si hanno le premesse, casi particolari, che danno valenza a una certa regola, che \`e appunto una generalizzazione dei casi particolari. Di tale valenza non ne siamo sicuri a priori e deve essere dimostrata matematicamente.\\
Questo ragionamento viene usato come approccio in \textit{Machine Learning}.\\\\
Esempio\\
REGOLA: "Socrate \`e un uomo."\\
CASO: "Socrate mor\`i."\\
RISULTATO: "Tutti gli uomini sono mortali."
\item \textbf{Ragionamento Abduttivo}: da ricondurre a \textit{Peirce} (1839-1914).\\
Qui il ragionamento viene invertito. Difatti viene potizziato che un'implicazione valga anche al contrario.\\\\
Esempio\\
REGOLA: "Tutti gli uomini sono mortali."\\
CASO: "Socrate mor\`i."\\
RISULTATO: "Socrate \`e un uomo".
\end{itemize}
\noindent
In \textit{Machine Learning} si perte da dati osservati (casi base) per dedurre una regola (generalizzazione). Questo permette di:
\begin{itemize}
\item Descrivere i dati;
\item Effettuare predizione sui nuovi casi che non ho ancora visto.
\end{itemize}
\noindent
\end{document}