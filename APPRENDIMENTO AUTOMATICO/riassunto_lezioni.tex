\documentclass[10pt,a4paper]{book}
\usepackage[italian]{babel}
\usepackage[utf8]{inputenc}
\usepackage{amsmath}
\usepackage{amsfonts}
\usepackage{amssymb}
\usepackage{graphicx}
\usepackage{float} 
\begin{document}
\chapter{Introduzione}

\section{Tipi di ragionamenti}
\begin{itemize}
\item \textbf{Ragionamento Deduttivo}: da ricondurre a \textit{Aristotele} (384 a.C.-322a.C.).\\
Da una regola e da un caso ottengo un risultato. \`E il ragionamento che permette la dimostrazione dei teoremi, ed \`e stata la base dei AI degli anni '90.\\\\
Esempio\\
REGOLA: "Tutti gli uomini sono mortali."\\
CASO: "Socrate \`e un uomo."\\
RISULTATO: "Socrate \`e mortale."
\item \textbf{Ragionamento Induttivo}: da ricondurre a \textit{Bacon} (1561-1626).\\
Avviene un'operazione di generalizzazione. Si hanno le premesse, casi particolari, che danno valenza a una certa regola, che \`e appunto una generalizzazione dei casi particolari. Di tale valenza non ne siamo sicuri a priori e deve essere dimostrata matematicamente.\\
Questo ragionamento viene usato come approccio in \textit{Machine Learning}.\\\\
Esempio\\
REGOLA: "Socrate \`e un uomo."\\
CASO: "Socrate mor\`i."\\
RISULTATO: "Tutti gli uomini sono mortali."
\item \textbf{Ragionamento Abduttivo}: da ricondurre a \textit{Peirce} (1839-1914).\\
Qui il ragionamento viene invertito. Difatti viene potizziato che un'implicazione valga anche al contrario.\\\\
Esempio\\
REGOLA: "Tutti gli uomini sono mortali."\\
CASO: "Socrate mor\`i."\\
RISULTATO: "Socrate \`e un uomo".
\end{itemize}
\noindent
In \textit{Machine Learning} si parte da dati osservati (casi base) per dedurre una regola (generalizzazione). Questo permette di:
\begin{itemize}
\item Descrivere i dati;
\item Effettuare predizione sui nuovi casi che non ho ancora visto.
\end{itemize}

\section{Che cosa \`e Machine Learning}
Domande chiave per la \textit{Machine Learning}:
\begin{enumerate}
\item Quanto e perch\`e \`e utile un approccio basato su tecniche di Apprendimento Automatico;
\item Come si pu\`o apprendere;
\item Si pu\`o veramente apprendere.
\end{enumerate}
\noindent
Per iniziare a parlare di apprendimento \`e fondamentale trattare il concetto di algoritmo. Ovvero insieme finito di passi elementari sequenziali, che porta all'ottenimento di un risultato per risolvere un problema. L'algoritmo \`e un concetto fondamentale per l'informatica in quanto permette di capire se un problema \`e calcolabile e rappresenta l'oggetto cardine durante lo sviluppo di un \textit{software} (sviluppo di codice che permette la risoluzione effettiva di un problema da parte di un calcolatore).\\
Un esempio semplice di algoritmo \`e una ricetta di cucina composta da ingredienti e procedimento.\\\\
Tuttavia non sempre \`e possibile dare dei criteri oggettivi in modo da realizzare un algoritmo. Ipotizziamo di voler realizzare qualcosa che permette di individuare quali \textit{e-mail}, che arrivano nella nostra casella di posta elettronica, sono SPAM. \`E possibile? E quali criteri di differenziazione dovrebbe controllare l'algoritmo? E se le \textit{e-mail} a seguito di questa contromisura mutassero?. In conclusione non sempre si pu\`o risolvere un problema per mezzo du un algoritmo. I motivi per cui questo accade sono i seguenti:
\begin{itemize}
\item Impossibilit\`a di formalizzare esattamente il problema;
\item Rumore e/o incertezza nel input/output;
\item Elevata complessit\`a della soluzione;
\item Inefficienza della soluzione;
\item Si conosce solo il problema, ma non si sa come fare a risolverlo.
\end{itemize}
\noindent
L'apprendimento, dal canto suo, \`e di primaria importanza, quando un sistema deve:
\begin{itemize}
\item Adattarsi all'ambiente in cui opera;
\item Migliorare la propria \textit{performance} nei confronti di un compito specifico;
\item Scoprire regolarit\`a e nuova conoscenza da dati empirici (ovvero non solo teorici, ma pratici).
\end{itemize}
I dati ormai sono abbondanti e presenti ovunque, questo ha permesso una crescita esponenziale della \textit{Machine Learning}, specialmente durante l'ultimo ventennio. Con formalismo la \textit{Machine Learning} \`e quella materia dove vengono studiati metodi per trasformare i dati in nuova conoscenza.\\
\`E importante per questo comprendere come esiste un processo stocastico (stato successivo non determinato dallo stato corrente, dove dunque \`e in auge la probabilit\`a) che spiega i dati che osserviamo, anche se non nei dettagli; e come l'apprendimento sia in grado di costruire delle buone approssimazioni, utili, su questo processo.\\
L'obiettivo finale dell'apprendimento, e della \textit{Machine Learning} in generale, \`e quello di definire dei criteri di \textit{performance} e di ottimizzarli utilizzando l'esperienza pregressa o i dati. I modelli di apprendimento si suddividono in modelli Preattivi che effettuano previsioni sul futuro e Descrittivi, che invece, permettono di ottenere nuova conoscenza.\\\\
Vi sono numerosi esempi applicativi di \textit{Machine Learning}. Il riconoscimento facciale, la \textit{Named Entity Recognition} che permette di identificare entit\`a in un una frase, la classificazione dei documenti da un insieme di \textit{topics}, i giochi e la profilazione del avversario, anche nel campo della bioinformatica per comprendere, per esempio, come reagir\`a un paziente a una certa terapia, e molti altri.\\
Di nota \`e \textit{Deep Blue} che nel 1997 \`e stata la prima macchina intelligente a riuscire a sconfiggere l'allora campione mondiale di scacchi, \textit{Kasparov}; e \textit{AphaGo} che nel 2016 riusc\`i a battere il campione di \textit{Go} \textit{Lee Sedal}.\\
Esistono anche le \textit{Generative Adversarial Learning} che sulla base di immagini di \textit{training} riescono a inventarsene di nuove. Questo dimostra come la \textit{Machine Learning} pu\`o assumere un comportamento creativo. O ancora esiste un sistema composto da Rete generatrice e discriminante, in cui la prima deve riuscire a comprendere se le istanza  sono create dalla Rete o da un essere umano e la seconda a crearne di verosimili.
\end{document}