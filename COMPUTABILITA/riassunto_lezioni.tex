\documentclass[10pt,a4paper]{book}
\usepackage[italian]{babel}
\usepackage[utf8]{inputenc}
\usepackage{amsmath}
\usepackage{amsfonts}
\usepackage{amssymb}
\begin{document}

\chapter{Introduzione}
\section{Argomenti}
Pensiamo a un punto lontano nel tempo 2150: necessit\`a di tenere traccia di $n$ numeri di telefono, risparmiando spazio scrivendoli per intero.\\
Idea di rubrica: non scrivere il numero interamente, ma un programma pi\`u breve del numero di telefono. Questa sembra una memorizzazione efficiente.\\
Tuttavia non \`e sempre vero:
\begin{equation}size(n) <= size(P)\end{equation}
Causato dall'esistenza dei numeri casuali, ovvero programmi di dimensione superiore al numero stesso.\\
I numeri casuali sono in numero infinto, inoltre non \`e possibile realizzare un programma che determini se un numero \`e casuale.\\\\
Conclusione: non tutto si pu\`o risolvere per mezzo di un calcolatore, questi problemi sono ben formalizzati. Alcuni esempi sono:
\begin{itemize}
\item \textit{Haiting Input} (problema della non terminazione di un programma dato uno specifico input);
\item Correttezza dei programmi.
\end{itemize}
L'obiettivo del corso \`e:
\begin{itemize}
\item Quali sono i problemi risolvibili con una procedura effettiva;
\item Cosa \`e una procedura effettiva;
\item Cosa vuol dire che un problema \`e risolto;
\item Problemi risolvibili e non risolvibili.
\end{itemize}
\noindent
Considereremo sempre che le risorse non sono un problema.

\section{Storicamente}
L'informatica inizia a trattare i problemi di computabilit\`a ancora prima di nascere.\\\\
\noindent
\textit{Dijkstra} parla di "informatica scienza delle procedure".\\
L'informatica \`e figlia della logica intesa come studio dei maccanismi del ragionamento (da dalle premesse si riesce a trarre delle conclusioni).\\
Si parla di schemi di ragionamento che portano a delle varit\`a sempre vere:\\
Ogni uomo \`e morale $\longrightarrow$ Socrate \`e un uomo $\longrightarrow$ Socrate \`e un uomo
\\\\
\textit{Lullus} cosruisce il primo esempio di meccanismo, la ruola lulliana. Crea frasi su cui l'uomo pu\`o ragionare.
\\\\
\textit{Leibniz} idea una lingua artificiale per mettere in relazione dei concetti (lingua per rappresentare le relazioni tra i pensieri umani). In questo modo viene favorita una lingua universale che permette l'apprendimento e lo sviluppo del sapere:
\begin{itemize}
\item \textit{Caracteristica Universalis}: linguaggio universale che permette di esprimere nozione e concetto. Usa simboli, nozioni di base e grammatica.
\item \textit{Calculus ratiocinator}: ragionamento attraverso una manipolazione di simboli.
\end{itemize}
\noindent
Questo permette di avere un metodo matematico che risolve le controversie. Un esempio pratico di questa idea \`e la \textit{Staffelwalze}, la prima macchina calcolatrice che computa le quattro operazioni aritmetiche.
\\\\
\textit{Boole} riprende le idee di \textit{Leibniz}. vuole interpretare con un'algebra (formalismo simbolico) i meccanismi della logica: algebra booleana che fornisce uno strumento di congiunzione tra i concetti. Per farlo tutto viene regolamentato da leggi $\in$ {0,1} che consentendo la creazione di leggi universalmente accettate.\\\\
\textit{Baddage} sviluppa  idealmente delle macchine \textit{general purpose} per il calcolo dei polinomi e a schede perforate programmabili con memoria e unit\`a aritmetica. Quest'ultima ha permesso lo sviluppo di ADA.\\\\
\textit{Frege} inventa la logica del primo ordine e si occupa di togliere la circolarit\`a presente all'interno della Logica di \textit{Boole}. Introduce i quantificatori esiste e per ogni. Tuttavia \textit{Russel} individua un paradosso nei suoi ragionamenti \{$x | x \notin x$\}.\\\\
\noindent
\textit{Hilbert} ricerca un sistema matematico formale composto da regole e assiomi, che si dimostri consistente e solido, senza alcun paradosso. In modo da ottenere un algoritmo in grado di definire se un teorema \`e conseguenza diretta dei suoi assiomi.\\\\
\textit{Godel} dimostra l'impossibilit\`a delle aspettative di \textit{Hilbert}. Esiste sempre qualcosa di vero impossibile da dimostrare. Inotre se T \`e assiomatizzabile non \`e possibile dimostrarne la consistenza interna.\\\\
\textit{Turing} definisce il concetto di algoritmo e si inventa una macchina programmabile con funzione calcolabile per dimostrare che non tutti i teoremi sono risolvibili. In questo modo \textit{Turing} dimostra l'esistenza di una macchina universale che funge da interprete (esecuzione di programmi sui dati di \textit{input}). Tuttavia la realizzazione del primo calcolatore programmabile fallisce a causa di limiti di risorse.\\\\
\textit{Von Neumann} realizza il primo progetto di macchina calcolatrice dall'idea della macchina di \textit{Turing} universale, realizzando il \textit{computer} digitale moderno.\\\\
\pagebreak

\chapter{Algoritmo e calcolabilit\`a}

"Gli algoritmi sono pochi, le cose da calcolare tante."\\\\
\noindent
Un algoritmo o procedura \`e una sequenza di passi elementari (meccanici, senza intelligenza) che permettono di ottenere un certo risultato.\\\\
Passi elementari: somma di due cifre.\\
$172+ 43 = 215$ 
\\
\`E perci\`o un procedimento guidato deterministico $f:\{input\} \longrightarrow \{output\}$ dove f \`e calcolata dall'algoritmo.\\\\
La funzione f \`e calcolabile se esiste un algoritmo che la calcola. Attenzione se esiste, non se la conosciamo.\\\\
Esempi di funzioni calcolabili:
\begin{itemize}
\item  $f(x,y) = x + y$ \\
La somma \`e calcolabile.
\item  $f(x,y) =
\begin{cases}
1 \verb! x primo! \\ 0 \verb! x non primi!
\end{cases}
$\\
L'individuazione dei numeri primi \`e calcolabile.

\item $f(x) = x\textsuperscript{mo}$\\
\`E difficile e ci si impiega tempo, vanno calcolati tutti mo, ma \`e calcolabile.
\item $g(n) = n\textsuperscript{ma}$ cifra di $\pi$\\
\`E difficile e ci si impiega tempo, vanno calcolati tutti mo, ma \`e calcolabile.
\item $h(n) =
\begin{cases}
1 \verb! se in ! \pi \verb! ci sono esattamente 5 cifre consecutive! \\ 0 \verb! altrimenti!
\end{cases}
$\\
Le cifre di $\pi$ vanno calcolate un p\`o alla volta. Se $\pi$ \`e normale (successioni con la medesima lunghezza che appaiono con la stessa frequenza) allora la costante \`e 1, altrimenti non lo so.
\item  $h'(n) =
\begin{cases}
1 \verb! se esistono almeno n cifre 5 consecutive in ! \pi \\ 0 \verb! altrimenti!
\end{cases}
$\\
Vale come sopra, se $\pi$ \`e normale allora la costante \`e 1, altrimenti non posso dire se \`e calcolabile.\\ Inoltre posso esprimere $k$ =  sup \{n $|$ n 5  consecutivi in $\pi$ \} =
\begin{enumerate}  
\item $k$ finito;
\item $\infty$ se $\pi$ normale.
\end{enumerate}
Ovvero:
\begin{enumerate}
\item  $h'(n) =
\begin{cases}
1 \verb! se n <= k (per ogni sottoinsieme inferiore)! \\ 0 \verb! altrimenti!
\end{cases}
$
\item  $h'(n) = 1  \verb! !\forall \verb!n! $
\end{enumerate}
\end{itemize}


\end{document}