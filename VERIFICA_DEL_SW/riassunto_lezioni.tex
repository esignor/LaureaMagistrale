\documentclass[10pt,                    % corpo del font principale
               a4paper,                 % carta A4
               twoside,                 % impagina per fronte-retro
               openright,               % inizio capitoli a destra
               english,                 
               italian,                 
               ]{book}
\usepackage[utf8]{inputenc}
\usepackage{amsmath}
\usepackage{amsfonts}
\usepackage{amssymb}
\usepackage[english, italian]{babel}
\usepackage{hyperref}
\begin{document}
\chapter{Introduzione}

\section{Analisi statica}
Analisi statica (alla SWE): analisi di un programma senza l'esecuzione a \textit{runtime}. Permette di ottenere informazioni sul programma stesso.
\\\\
L'obiettivo del corso di Verifica del \textit{software} comprende l'utilizzo della correttezza per comprendere l'esecuzione di un programma a \textit{runtime}. Permette di ottenere la piena garanzia di correttezza del programma.\\
La verifica pu\`o venire fatta per qualunque versione del codice sorgente (\textit{bytecode} o altro). Viene svolta con analizzatori di codice o moduli. Quest'ultimi effettuano una verifica progressiva, parallela alla scrittura.
\\
Polyspace: strumento di analisi di codice statico, dimostra l'esistenza di errori a \textit{runtime} critici. Verifica codice C, C++ o Ada.\\
Un problema esistente \`e la mancata scalabilit\`a della verifica.\\
Quando si scrive un analizzatore \`e bene utilizzare un linguaggio di programmazione robusto (come Ada).
Tuttavia la quasi assenza di persone competenti in tale ambito incide sulla difficolt\`a di risolvere i problemi legati agli analizzatori.\\\\
Alcuni esempi di analizzatori:
\begin{itemize}
\item Interproc \`e un analizzatore accademico che inferisce invarianti;
\item Jandom \`e un analizzatore Java, scritto in Scala (ovvero Java funzionale avanzato).
\end{itemize}
\noindent
Di recente \`e l'impiego degli analizzatori per gli algoritmi di \textit{Machine Learning}.

\section{Motivazioni}
Le motivazioni che portano allo studio dell'analisi statica:
\begin{enumerate}
\item Fallimenti \textit{software}: caso di Ariane 5.\\
Il razzo una volta lanciato in aria si autodistrugge.\\
Si era verificato un \textit{software error}, conversione tra virgola mobile a intero, con perdita d'informazione, che ha lanciato un'eccezione non catturata causando la chiusura di tutti i programmi del razzo. L'ultimo programma eseguito \`e stato l'autodistruzione.
\item \textit{Meltdown} e \textit{Spectre}: i processori moderni usano l'esecuzione speculativa (tecnica di otimizzazione. L'eleaboratore esegue operazioni necessarie forse solo in un secondo tempo. \url{https://it.wikipedia.org/wiki/Esecuzione_speculativa.}) per velocizzare il lavoro. Un \textit{team} di ricercatori ha su tale tecnica individuato un \textit{bug} di sicurezza, causato delle \textit{miss prediction} che lasciavano dati sensibili all'inteno delle \textit{cache}, innescando eventuali attacchi malevoli.
L'Intel, precedentemente la pubblicazione della ricerca, avvisata dal \textit{team}, ha mitigato il problema in modo \textit{hardware}. Un modo per impedire la nascita di questi \textit{bug} \`e svolgere gi\`a durante lo sviluppo del codice l'analisi statica. 
\end{enumerate}
\noindent
Una buona tecnica, per punti, che permette di prevenire fallimenti del codice \`e la seguente:
\begin{itemize}
\item Scegliere un buon linguaggio;
\item Svolgere progettazione;
\item Effettuare \textit{code testing};
\item Utilizzare un metodo formale di analisi statica. Questo permette la totale garanzia della correttezza del codice prodotto.
\end{itemize}
\newpage

\chapter{Programmazione Semantica}
La semantica \`e un asse fondamentale della \textit{Programming Language Theory} e si occupa di valutare una sequenza di simboli con regole (sintassi) e i meccanismi di grammatica in modo da mostrare il significato del calcolo. In sostanza permette di realizzare un modello.\\
Il caso di \textit{parser} che rigetta il programma \`e un problema di sintassi, non di nostra competenza.\\\\
\noindent
\verb! if 1=1 then S1 else S2!\\
semanticamente equivale a scrivere $S1$.\\\\\\
Perch\`e fare semantica?
\begin{itemize}
\item Per dare una specifica su un linguaggio;
\item Come supporto del \textit{testing};
\item Permette di avere le basi per un prototipo (per interpreti e compilatori).
\end{itemize}
\noindent
\textit{CompCert} \`e un compilatore per C, con impatto pratico e industriale certificato. Il codice che produce (oggetto) \`e equivalente al codice sorgente. Per questo c'\`e un modello sia per il codice oggetto che per il codice sorgente. CompCert usa il dimostratore Coq, che permette di verificare teoremi.\\
Il codice generato da CompCert \`e vicino al GCC, ritenuto il pi\`u efficiente ed \`e a licenza per usi commerciali e \textit{open-source} se per usi personali.\\\\
Il linguaggio che useremo un questo corso \`e \textit{Turing} completo e non impiegeremo il tipo puntatore in modo da non fare analisi statica all'interno della memoria.\\\\


Perch\`e utilizzare un modello?
\begin{itemize}
\item Algo160 usa il linguaggio naturale, tuttavia in questo modo la semantica diventa incomprensibile. La scelta pi\`u adatta \`e scrivere un modello con una funzione che spiega, per esempio, che cosa sono i conflitti.
\item Nello \textit{standard} del C++ la descrizione richiama un modello. La scelta migliore \`e quella di scriverlo in modo formale, per rendere pi\`u chiare le cose.
\end{itemize}
\noindent
\textbf{Semantica Operazionale}: da un significato operazionale, descrive ogni operazione del programma nel dettaglio. Si divide in due tipi:
\begin{itemize}
\item Semantica Naturale;
\item Semantica Operazionale Strutturale: \`e un interprete, si occupa di dare all'interpretatore le istruzioni senza alcuna ambiguit\`a.
\end{itemize}
\noindent
\textbf{Semantica Denotazionale}: in questo caso si parla di funzione da stato a stato, questo per\`o non significa sempre che la funzione termina sempre, pu\`o accadere che la funzione non sia definita.  Pu\`o descrivere la terminazione con i possibili stati raggiunti dalle variabili. In questo caso i tipi sono:
\begin{itemize}
\item Stile Semantico Diretto
\item Stile Semantico di Continuazione: con \textit{jump} e chiamate di funzione.
\end{itemize}
\noindent
\textbf{Semantica Assiomatica} \`e la semantica degli invarianti. Specifica il significato di un programma rispetto a pre e postcondizioni.\\
Se so per certo che il programma termina, e si ha la prova di terminazione, si parla di Correttezza Totale, altrimenti ci dobbiamo limitare alla Correttazza Parziale. \\\\
\noindent
Quale sematica scegliere? Dipende dall'obiettivo.
Esistono diversi criteri di selezione della sematica: il costrutto del linguaggio (imperativo, funzionale, concorrente/parallelo ..), per che cosa si usa la semantica (per creare un prototipo, analisi di programma, verifica di un programma, comprendere un linguaggio, ..).
\\\\
\noindent
Durante il corso faremo uso  del \textit{While language}\\
\noindent
\verb! S ::= x :=a | skip | S1;S2!\\
\verb!             | if b then S1 else S2!\\
\verb!             | while b do S!\\
\verb!             | repeat S until b!\\
e della notazione BNF che permette di scrivere in modo compatto una grammatica.
\\\\
\textbf{Categorie Sintattiche con While language}
\begin{itemize}
\item numeri \\
n $\in$ Num 

\item variabili
x $\in$ Var\\
Useremo il tipo numerico intero e nei valori letterali non specificheremo la sintassi.
\item arithmetic expression
a $\in$ Aexp
\verb! a ::= n | x | a1+a2!\\
\verb!             | a1*a2 | a1-a2!\\
Non c'\`e la divisione per 0, perch\`e provoca un errore \textit{run-time} che inificia la semantica e non la sintassi. Questo per\`o deve essere presente durante l'analisi statica perch\`e \`e proprio il fenomeno che vogliamo evitare.
\item boolean
a $\in$ Bexp
\verb! b ::= true | false | a1=a2!\\
\verb!             | a1<=a2 |! $\lnot$ \verb!b | b1! $\lor$ \verb!b2!\\
\item statements
S $\in$ Stm
\verb! S ::= x :=a | skip | S1;S2!\\
\verb!             | if b then S1 else S2!\\
\verb!             | while b do S!\\


\end{itemize}
\textbf{Categorie Semantiche}
\begin{itemize}
\item Modello per il tipo intero
N = \{..-2,-1,0,1,2,..\} $\in$ Z\\
T = \{tt, ff\} semantica dei valori di verit\`a\\
La memoria memorizza lo stato del calcolo, dunque le variabili sono locazioni di memoria.\\
State =  Var $\rightarrow$ N  (variabili che occorrono in un programma) \\
\textit{lookup}  sx\\
\textit{upadate} $s'x$ = s[y $\rightarrow$ v]\\
$s'x =
\begin{cases}
sx\verb! x! \ne \verb!y! \\ v \verb! if x = y!
\end{cases}
$\\

\end{itemize}

\end{document}