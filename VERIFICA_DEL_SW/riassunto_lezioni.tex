\documentclass[10pt,                    % corpo del font principale
               a4paper,                 % carta A4
               twoside,                 % impagina per fronte-retro
               openright,               % inizio capitoli a destra
               english,                 
               italian,                 
               ]{book}
\usepackage[utf8]{inputenc}
\usepackage{amsmath}
\usepackage{amsfonts}
\usepackage{amssymb}
\usepackage[english, italian]{babel}
\usepackage{hyperref}
\begin{document}
\chapter{Introduzione}

\section{Analisi statica}
Analisi statica (alla SWE): analisi di un programma senza l'esecuzione a \textit{runtime}. Permette di ottenere informazioni sul programma stesso.
\\\\
L'obiettivo del corso di Verifica del \textit{software} comprende l'utilizzo della correttezza per comprendere l'esecuzione di un programma a \textit{runtime}. Permette di ottenere la piena garanzia di correttezza del programma.\\
La verifica pu\`o venire fatta per qualunque versione del codice sorgente (\textit{bytecode} o altro). Viene svolta con analizzatori di codice o moduli. Quest'ultimi effettuano una verifica progressiva, parallela alla scrittura.
\\
Polyspace: strumento di analisi di codice statico, dimostra l'esistenza di errori a \textit{runtime} critici. Verifica codice C, C++ o Ada.\\
Un problema esistente \`e la mancata scalabilit\`a della verifica.\\
Quando si scrive un analizzatore \`e bene utilizzare un linguaggio di programmazione robusto (come Ada).
Tuttavia la quasi assenza di persone competenti in tale ambito incide sulla difficolt\`a di risolvere i problemi legati agli analizzatori.\\\\
Alcuni esempi di analizzatori:
\begin{itemize}
\item Interproc \`e un analizzatore accademico che inferisce invarianti;
\item Jandom \`e un analizzatore Java, scritto in Scala (ovvero Java funzionale avanzato).
\end{itemize}
\noindent
Di recente \`e l'impiego degli analizzatori per gli algoritmi di \textit{Machine Learning}.

\section{Motivazioni}
Le motivazioni che portano allo studio dell'analisi statica:
\begin{enumerate}
\item Fallimenti \textit{software}: caso di Ariane 5.\\
Il razzo una volta lanciato in aria si autodistrugge.\\
Si era verificato un \textit{software error}, conversione tra virgola mobile a intero, con perdita d'informazione, che ha lanciato un'eccezione non catturata causando la chiusura di tutti i programmi del razzo. L'ultimo programma eseguito \`e stato l'autodistruzione.
\item \textit{Meltdown} e \textit{Spectre}: i processori moderni usano l'esecuzione speculativa (tecnica di otimizzazione. L'eleaboratore esegue operazioni necessarie forse solo in un secondo tempo. \url{https://it.wikipedia.org/wiki/Esecuzione_speculativa.}) per velocizzare il lavoro. Un \textit{team} di ricercatori ha su tale tecnica individuato un \textit{bug} di sicurezza, causato delle \textit{miss prediction} che lasciavano dati sensibili all'inteno delle \textit{cache}, innescando eventuali \textit{time attact}.
L'Intel, precedentemente la pubblicazione della ricerca, avvisata dal \textit{team}, ha mitigato il problema in modo \textit{hardware}. Un modo per impedire la nascita di questi \textit{bug} \`e svolgere gi\`a durante lo sviluppo del codice l'analisi statica. 
\end{enumerate}
\noindent
Una buona tecnica, per punti, che permette di prevenire fallimenti del codice \`e la seguente:
\begin{itemize}
\item Scegliere un buon linguaggio;
\item Svolgere progettazione;
\item Effettuare \textit{code testing};
\item Utilizzare un metodo formale di analisi statica. Questo permette la totale garanzia della correttezza del codice prodotto.
\end{itemize}

\end{document}